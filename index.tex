% Options for packages loaded elsewhere
\PassOptionsToPackage{unicode}{hyperref}
\PassOptionsToPackage{hyphens}{url}
\PassOptionsToPackage{dvipsnames,svgnames,x11names}{xcolor}
%
\documentclass[
  letterpaper,
  DIV=11,
  numbers=noendperiod]{scrreprt}

\usepackage{amsmath,amssymb}
\usepackage{lmodern}
\usepackage{iftex}
\ifPDFTeX
  \usepackage[T1]{fontenc}
  \usepackage[utf8]{inputenc}
  \usepackage{textcomp} % provide euro and other symbols
\else % if luatex or xetex
  \usepackage{unicode-math}
  \defaultfontfeatures{Scale=MatchLowercase}
  \defaultfontfeatures[\rmfamily]{Ligatures=TeX,Scale=1}
\fi
% Use upquote if available, for straight quotes in verbatim environments
\IfFileExists{upquote.sty}{\usepackage{upquote}}{}
\IfFileExists{microtype.sty}{% use microtype if available
  \usepackage[]{microtype}
  \UseMicrotypeSet[protrusion]{basicmath} % disable protrusion for tt fonts
}{}
\makeatletter
\@ifundefined{KOMAClassName}{% if non-KOMA class
  \IfFileExists{parskip.sty}{%
    \usepackage{parskip}
  }{% else
    \setlength{\parindent}{0pt}
    \setlength{\parskip}{6pt plus 2pt minus 1pt}}
}{% if KOMA class
  \KOMAoptions{parskip=half}}
\makeatother
\usepackage{xcolor}
\setlength{\emergencystretch}{3em} % prevent overfull lines
\setcounter{secnumdepth}{5}
% Make \paragraph and \subparagraph free-standing
\ifx\paragraph\undefined\else
  \let\oldparagraph\paragraph
  \renewcommand{\paragraph}[1]{\oldparagraph{#1}\mbox{}}
\fi
\ifx\subparagraph\undefined\else
  \let\oldsubparagraph\subparagraph
  \renewcommand{\subparagraph}[1]{\oldsubparagraph{#1}\mbox{}}
\fi

\usepackage{color}
\usepackage{fancyvrb}
\newcommand{\VerbBar}{|}
\newcommand{\VERB}{\Verb[commandchars=\\\{\}]}
\DefineVerbatimEnvironment{Highlighting}{Verbatim}{commandchars=\\\{\}}
% Add ',fontsize=\small' for more characters per line
\usepackage{framed}
\definecolor{shadecolor}{RGB}{241,243,245}
\newenvironment{Shaded}{\begin{snugshade}}{\end{snugshade}}
\newcommand{\AlertTok}[1]{\textcolor[rgb]{0.68,0.00,0.00}{#1}}
\newcommand{\AnnotationTok}[1]{\textcolor[rgb]{0.37,0.37,0.37}{#1}}
\newcommand{\AttributeTok}[1]{\textcolor[rgb]{0.40,0.45,0.13}{#1}}
\newcommand{\BaseNTok}[1]{\textcolor[rgb]{0.68,0.00,0.00}{#1}}
\newcommand{\BuiltInTok}[1]{\textcolor[rgb]{0.00,0.23,0.31}{#1}}
\newcommand{\CharTok}[1]{\textcolor[rgb]{0.13,0.47,0.30}{#1}}
\newcommand{\CommentTok}[1]{\textcolor[rgb]{0.37,0.37,0.37}{#1}}
\newcommand{\CommentVarTok}[1]{\textcolor[rgb]{0.37,0.37,0.37}{\textit{#1}}}
\newcommand{\ConstantTok}[1]{\textcolor[rgb]{0.56,0.35,0.01}{#1}}
\newcommand{\ControlFlowTok}[1]{\textcolor[rgb]{0.00,0.23,0.31}{#1}}
\newcommand{\DataTypeTok}[1]{\textcolor[rgb]{0.68,0.00,0.00}{#1}}
\newcommand{\DecValTok}[1]{\textcolor[rgb]{0.68,0.00,0.00}{#1}}
\newcommand{\DocumentationTok}[1]{\textcolor[rgb]{0.37,0.37,0.37}{\textit{#1}}}
\newcommand{\ErrorTok}[1]{\textcolor[rgb]{0.68,0.00,0.00}{#1}}
\newcommand{\ExtensionTok}[1]{\textcolor[rgb]{0.00,0.23,0.31}{#1}}
\newcommand{\FloatTok}[1]{\textcolor[rgb]{0.68,0.00,0.00}{#1}}
\newcommand{\FunctionTok}[1]{\textcolor[rgb]{0.28,0.35,0.67}{#1}}
\newcommand{\ImportTok}[1]{\textcolor[rgb]{0.00,0.46,0.62}{#1}}
\newcommand{\InformationTok}[1]{\textcolor[rgb]{0.37,0.37,0.37}{#1}}
\newcommand{\KeywordTok}[1]{\textcolor[rgb]{0.00,0.23,0.31}{#1}}
\newcommand{\NormalTok}[1]{\textcolor[rgb]{0.00,0.23,0.31}{#1}}
\newcommand{\OperatorTok}[1]{\textcolor[rgb]{0.37,0.37,0.37}{#1}}
\newcommand{\OtherTok}[1]{\textcolor[rgb]{0.00,0.23,0.31}{#1}}
\newcommand{\PreprocessorTok}[1]{\textcolor[rgb]{0.68,0.00,0.00}{#1}}
\newcommand{\RegionMarkerTok}[1]{\textcolor[rgb]{0.00,0.23,0.31}{#1}}
\newcommand{\SpecialCharTok}[1]{\textcolor[rgb]{0.37,0.37,0.37}{#1}}
\newcommand{\SpecialStringTok}[1]{\textcolor[rgb]{0.13,0.47,0.30}{#1}}
\newcommand{\StringTok}[1]{\textcolor[rgb]{0.13,0.47,0.30}{#1}}
\newcommand{\VariableTok}[1]{\textcolor[rgb]{0.07,0.07,0.07}{#1}}
\newcommand{\VerbatimStringTok}[1]{\textcolor[rgb]{0.13,0.47,0.30}{#1}}
\newcommand{\WarningTok}[1]{\textcolor[rgb]{0.37,0.37,0.37}{\textit{#1}}}

\providecommand{\tightlist}{%
  \setlength{\itemsep}{0pt}\setlength{\parskip}{0pt}}\usepackage{longtable,booktabs,array}
\usepackage{calc} % for calculating minipage widths
% Correct order of tables after \paragraph or \subparagraph
\usepackage{etoolbox}
\makeatletter
\patchcmd\longtable{\par}{\if@noskipsec\mbox{}\fi\par}{}{}
\makeatother
% Allow footnotes in longtable head/foot
\IfFileExists{footnotehyper.sty}{\usepackage{footnotehyper}}{\usepackage{footnote}}
\makesavenoteenv{longtable}
\usepackage{graphicx}
\makeatletter
\def\maxwidth{\ifdim\Gin@nat@width>\linewidth\linewidth\else\Gin@nat@width\fi}
\def\maxheight{\ifdim\Gin@nat@height>\textheight\textheight\else\Gin@nat@height\fi}
\makeatother
% Scale images if necessary, so that they will not overflow the page
% margins by default, and it is still possible to overwrite the defaults
% using explicit options in \includegraphics[width, height, ...]{}
\setkeys{Gin}{width=\maxwidth,height=\maxheight,keepaspectratio}
% Set default figure placement to htbp
\makeatletter
\def\fps@figure{htbp}
\makeatother
\newlength{\cslhangindent}
\setlength{\cslhangindent}{1.5em}
\newlength{\csllabelwidth}
\setlength{\csllabelwidth}{3em}
\newlength{\cslentryspacingunit} % times entry-spacing
\setlength{\cslentryspacingunit}{\parskip}
\newenvironment{CSLReferences}[2] % #1 hanging-ident, #2 entry spacing
 {% don't indent paragraphs
  \setlength{\parindent}{0pt}
  % turn on hanging indent if param 1 is 1
  \ifodd #1
  \let\oldpar\par
  \def\par{\hangindent=\cslhangindent\oldpar}
  \fi
  % set entry spacing
  \setlength{\parskip}{#2\cslentryspacingunit}
 }%
 {}
\usepackage{calc}
\newcommand{\CSLBlock}[1]{#1\hfill\break}
\newcommand{\CSLLeftMargin}[1]{\parbox[t]{\csllabelwidth}{#1}}
\newcommand{\CSLRightInline}[1]{\parbox[t]{\linewidth - \csllabelwidth}{#1}\break}
\newcommand{\CSLIndent}[1]{\hspace{\cslhangindent}#1}

\KOMAoption{captions}{tableheading}
\makeatletter
\@ifpackageloaded{tcolorbox}{}{\usepackage[many]{tcolorbox}}
\@ifpackageloaded{fontawesome5}{}{\usepackage{fontawesome5}}
\definecolor{quarto-callout-color}{HTML}{909090}
\definecolor{quarto-callout-note-color}{HTML}{0758E5}
\definecolor{quarto-callout-important-color}{HTML}{CC1914}
\definecolor{quarto-callout-warning-color}{HTML}{EB9113}
\definecolor{quarto-callout-tip-color}{HTML}{00A047}
\definecolor{quarto-callout-caution-color}{HTML}{FC5300}
\definecolor{quarto-callout-color-frame}{HTML}{acacac}
\definecolor{quarto-callout-note-color-frame}{HTML}{4582ec}
\definecolor{quarto-callout-important-color-frame}{HTML}{d9534f}
\definecolor{quarto-callout-warning-color-frame}{HTML}{f0ad4e}
\definecolor{quarto-callout-tip-color-frame}{HTML}{02b875}
\definecolor{quarto-callout-caution-color-frame}{HTML}{fd7e14}
\makeatother
\makeatletter
\makeatother
\makeatletter
\@ifpackageloaded{bookmark}{}{\usepackage{bookmark}}
\makeatother
\makeatletter
\@ifpackageloaded{caption}{}{\usepackage{caption}}
\AtBeginDocument{%
\ifdefined\contentsname
  \renewcommand*\contentsname{Зміст}
\else
  \newcommand\contentsname{Зміст}
\fi
\ifdefined\listfigurename
  \renewcommand*\listfigurename{List of Figures}
\else
  \newcommand\listfigurename{List of Figures}
\fi
\ifdefined\listtablename
  \renewcommand*\listtablename{List of Tables}
\else
  \newcommand\listtablename{List of Tables}
\fi
\ifdefined\figurename
  \renewcommand*\figurename{Figure}
\else
  \newcommand\figurename{Figure}
\fi
\ifdefined\tablename
  \renewcommand*\tablename{Table}
\else
  \newcommand\tablename{Table}
\fi
}
\@ifpackageloaded{float}{}{\usepackage{float}}
\floatstyle{ruled}
\@ifundefined{c@chapter}{\newfloat{codelisting}{h}{lop}}{\newfloat{codelisting}{h}{lop}[chapter]}
\floatname{codelisting}{Listing}
\newcommand*\listoflistings{\listof{codelisting}{List of Listings}}
\makeatother
\makeatletter
\@ifpackageloaded{caption}{}{\usepackage{caption}}
\@ifpackageloaded{subcaption}{}{\usepackage{subcaption}}
\makeatother
\makeatletter
\@ifpackageloaded{tcolorbox}{}{\usepackage[many]{tcolorbox}}
\makeatother
\makeatletter
\@ifundefined{shadecolor}{\definecolor{shadecolor}{rgb}{.97, .97, .97}}
\makeatother
\makeatletter
\makeatother
\ifLuaTeX
  \usepackage{selnolig}  % disable illegal ligatures
\fi
\IfFileExists{bookmark.sty}{\usepackage{bookmark}}{\usepackage{hyperref}}
\IfFileExists{xurl.sty}{\usepackage{xurl}}{} % add URL line breaks if available
\urlstyle{same} % disable monospaced font for URLs
\hypersetup{
  pdftitle={Вступ до програмування в R},
  pdfauthor={Юрій Клебан},
  colorlinks=true,
  linkcolor={blue},
  filecolor={Maroon},
  citecolor={Blue},
  urlcolor={Blue},
  pdfcreator={LaTeX via pandoc}}

\title{Вступ до програмування в R}
\usepackage{etoolbox}
\makeatletter
\providecommand{\subtitle}[1]{% add subtitle to \maketitle
  \apptocmd{\@title}{\par {\large #1 \par}}{}{}
}
\makeatother
\subtitle{Матеріали для студентів спеціальності економічна кібернетика,
1 курс}
\author{Юрій Клебан}
\date{12/05/2022}

\begin{document}
\maketitle
\ifdefined\Shaded\renewenvironment{Shaded}{\begin{tcolorbox}[breakable, interior hidden, boxrule=0pt, enhanced, borderline west={3pt}{0pt}{shadecolor}, sharp corners, frame hidden]}{\end{tcolorbox}}\fi

\renewcommand*\contentsname{Зміст}
{
\hypersetup{linkcolor=}
\setcounter{tocdepth}{2}
\tableofcontents
}
\bookmarksetup{startatroot}

\hypertarget{ux43fux440ux43e-ux43aux443ux440ux441}{%
\chapter*{Про курс}\label{ux43fux440ux43e-ux43aux443ux440ux441}}
\addcontentsline{toc}{chapter}{Про курс}

Юрій Клебан\\
30.10.2022

\hfill\break

\begin{figure}

{\centering 

\href{https://www.python.org/}{\includegraphics{http://ForTheBadge.com/images/badges/made-with-python.svg}}

}

\caption{forthebadge made-with-python}

\end{figure}

\begin{figure}

{\centering 

\href{https://www.python.org/}{\includegraphics{http://ForTheBadge.com/images/badges/made-with-python.svg}}

}

\caption{forthebadge made-with-r}

\end{figure}

\begin{figure}

{\centering 

\href{http://commonmark.org}{\includegraphics{https://img.shields.io/badge/Made\%20with-Markdown-1f425f.svg}}

}

\caption{made-with-md}

\end{figure}

\begin{figure}

{\centering 

\href{https://www.mathjax.org/}{\includegraphics{https://img.shields.io/badge/Made\%20with-MathJax-1f425f.svg}}

}

\caption{made-with-mathjax}

\end{figure}

\begin{figure}

{\centering 

\href{https://GitHub.com/Naereen/StrapDown.js/commit/}{\includegraphics{https://badgen.net/github/commits/Naereen/Strapdown.js}}

}

\caption{GitHub commits}

\end{figure}

(n.d.)

Матеріали підготовлені для читання курсу \textbf{\emph{``Вступ до
програмування в R''}} \texttt{{[}05.250{]}} для студентів 1-го року
навчання, спеціальності економічна кібернетика.

\hypertarget{ux43eux43fux438ux441-ux43dux430ux432ux447ux430ux43bux44cux43dux43eux457-ux434ux438ux441ux446ux438ux43fux43bux456ux43dux438}{%
\section*{Опис навчальної
дисципліни}\label{ux43eux43fux438ux441-ux43dux430ux432ux447ux430ux43bux44cux43dux43eux457-ux434ux438ux441ux446ux438ux43fux43bux456ux43dux438}}
\addcontentsline{toc}{section}{Опис навчальної дисципліни}

Навчальна дисципліна спрямована на вивчення основ практичного
застосування популярної мови R для проведення статистичних досліджень в
економіці.

У процесі вивчення курсу розглядаються теми, що стосуються теоретичних
основ та практичної реалізації алгоритмів завантаження, підготовки та
обробки економічних даних.

Місце навчальної дисципліни у підготовці здобувачів: програмні
результати дисципліни використовуються під час вивчення таких навчальних
дисциплін: ``Алгоритми та структури даних'', ``Аналіз даних в R'',
``Прикладне математичне моделювання в R''. Закріплення на практиці
здобутих програмних результатів відбувається під час проходження
Навчальної практики з курсу «Економіко-математичне моделювання».

\hypertarget{ux43cux435ux442ux430-ux434ux438ux441ux446ux438ux43fux43bux456ux43dux438}{%
\section*{Мета
дисципліни}\label{ux43cux435ux442ux430-ux434ux438ux441ux446ux438ux43fux43bux456ux43dux438}}
\addcontentsline{toc}{section}{Мета дисципліни}

Мета навчальної дисципліни -- формування у студентів теоретичних знань
та практичних навичок використання мови програмування R для роботи з
даними та базовими структурами мови (типи даних, розгалуження, цикли,
функції).

\hypertarget{ux434ux43eux442ux440ux438ux43cux430ux43dux43dux44f-ux43fux440ux438ux43dux446ux438ux43fux456ux432-ux434ux43eux431ux440ux43eux447ux435ux441ux43dux43eux441ux442ux456}{%
\section*{Дотримання принципів
доброчесності}\label{ux434ux43eux442ux440ux438ux43cux430ux43dux43dux44f-ux43fux440ux438ux43dux446ux438ux43fux456ux432-ux434ux43eux431ux440ux43eux447ux435ux441ux43dux43eux441ux442ux456}}
\addcontentsline{toc}{section}{Дотримання принципів доброчесності}

Викладач та слухач цього курсу, як очікується, повинні дотримуватися
Кодексу академічної доброчесності університету:

\begin{itemize}
\item[$\boxtimes$]
  будь-яка робота, подана здобувачем протягом курсу, має бути його
  власною роботою здобувача; не вдаватися до кроків, що можуть нечесно
  покращити Ваші результати чи погіршити/покращити результати інших
  здобувачів;
\item[$\boxtimes$]
  якщо буде виявлено ознаки плагіату або іншої недобросовісної
  академічної поведінки, то студент буде позбавлений можливості отримати
  передбачені бали за завдання;
\item[$\boxtimes$]
  не публікувати у відкритому доступі відповіді на запитання, що
  використовуються в рамках курсу для оцінювання знань здобувачів;
\item[$\boxtimes$]
  під час фінальних видів контролю необхідно працювати самостійно; не
  дозволяється говорити або обговорювати, а також не можна копіювати
  документи, використовувати електронні засоби отримання інформації.
\end{itemize}

Порушення академічної доброчесності під час виконання контрольних
завдань призведе до втрати балів або вживання заходів, які передбачені
Кодексу академічної доброчесності НаУОА.

\hypertarget{ux437ux43cux456ux441ux442}{%
\section*{Зміст}\label{ux437ux43cux456ux441ux442}}
\addcontentsline{toc}{section}{Зміст}

\begin{Shaded}
\begin{Highlighting}[]

\end{Highlighting}
\end{Shaded}

\hypertarget{ux437ux43cux456ux441ux442-1}{%
\section*{Зміст}\label{ux437ux43cux456ux441ux442-1}}
\addcontentsline{toc}{section}{Зміст}

\protect\hyperlink{ux43fux440ux43e-ux43aux443ux440ux441}{Про курс}

\protect\hyperlink{ux432ux441ux442ux443ux43f}{Вступ}

\protect\hyperlink{ux441ux43fux438ux441ux43eux43a-ux432ux438ux43aux43eux440ux438ux441ux442ux430ux43dux438ux445-ux434ux436ux435ux440ux435ux43b}{Список
використаних джерел}

\begin{center}\rule{0.5\linewidth}{0.5pt}\end{center}

\hypertarget{ux43fux456ux434ux442ux440ux438ux43cux43aux430-ux43aux443ux440ux441ux443}{%
\section*{Підтримка
курсу}\label{ux43fux456ux434ux442ux440ux438ux43cux43aux430-ux43aux443ux440ux441ux443}}
\addcontentsline{toc}{section}{Підтримка курсу}

Курс створено у межах проекту ``Підготовка, обробка та ефективне
використання даних для наукових досліджень (на основі R)'', що підтримує
Європейський союз за програмою
\href{https://houseofeurope.org.ua/}{House of Europe}.

logo house of europe / logo moving forvard together

\begin{center}\rule{0.5\linewidth}{0.5pt}\end{center}

\begin{tcolorbox}[enhanced jigsaw, arc=.35mm, colback=white, rightrule=.15mm, breakable, colframe=quarto-callout-note-color-frame, toprule=.15mm, leftrule=.75mm, bottomrule=.15mm, left=2mm, opacityback=0]
\begin{minipage}[t]{5.5mm}
\textcolor{quarto-callout-note-color}{\faInfo}
\end{minipage}%
\begin{minipage}[t]{\textwidth - 5.5mm}

Матеріали курсу створені з використанням ряду технологій:

\begin{itemize}
\item[$\boxtimes$]
  \href{https://www.r-project.org}{\texttt{R} Language} - безкоштована
  мова програмування для виконання досліджень у сфері статистики,
  машинного навчання та візуалізацї результатів.
\item[$\boxtimes$]
  \href{https://quarto.org}{\texttt{Quarto} Book} - система для
  публікації наукових та технічних текстів з відкритим кодом
  (\texttt{R}/\texttt{Python}/\texttt{Julia}/\texttt{Observable}).
\item[$\boxtimes$]
  \href{https://github.com/jupyterlab/jupyterlab}{\texttt{JupyterLab}} -
  середовище розробки на основі
  \href{https://jupyter.org/}{\texttt{Jupyter\ Notebook}}.
  \texttt{JupyterLab} є розширеним веб-інтерфейсом для роботи з
  ноутбуками.
\item[$\boxtimes$]
  \texttt{Git}/\texttt{Github} - система контролю версій та, відповідно,
  сервіс для організації зберігання коду, а також публікації статичних
  сторінок.
\end{itemize}

\end{minipage}%
\end{tcolorbox}

\bookmarksetup{startatroot}

\hypertarget{ux432ux441ux442ux443ux43f}{%
\chapter*{Вступ}\label{ux432ux441ux442ux443ux43f}}
\addcontentsline{toc}{chapter}{Вступ}

Юрій Клебан\\
30.10.2022

\hfill\break

Фахівці спеціальності економічна кібернетика у майбутньому працюватимуть
з великими масивами даних, що накопичуються велизними темпами у даний
момент і збиралися у попередні дисятиліття. Підготовка, обробка і
трансформація даних у зручний формат прийняття рішень забирає все більше
часу, а звичні рашіне інструменти аналізу даних, як наприклад,
\texttt{Microsoft\ Excel} не мають достатньо вбудованих можливостей для
виконнання задач бізнесу.

На даний час існує велика кількість мов програмування, що інтегруються у
суспільні сфери діяльності людини та роботи технічних систем:
біоінформатика, а також економіка та бізнес.

Однією з мов програмування, що отримали широке поишення серед
економістів-науковців, аналітиків та практиків математичного моделювання
(\texttt{machine\ learning}) є мова програмування \texttt{(R)}(n.d.).
Свою популярність ця мова програмування здобула завдяки простоті у
використанні, доступності (безкоштовні як базові компоненти для
написання коду, так і середовища розробки), розширюваності (кожен
розробник має можливість створювати власні пакети та публікувати їх у
відкритому доступі).

Основними задачами курсу ``Вступ до прграмування в R'' є ознайомлення
студентів з базовми конструкціями мови програмування R, вивчення
способів роботи з найпоширенішими типами даних,

\begin{center}\rule{0.5\linewidth}{0.5pt}\end{center}

\part{Основи роботи в R}

\hypertarget{ux449ux43e-ux442ux430ux43aux435-r}{%
\chapter{Що таке R?}\label{ux449ux43e-ux442ux430ux43aux435-r}}

Юрій Клебан\\
30/11/2022

\hfill\break

R є поширеною мовою програмування для роботи з даними
(\texttt{DataScience}) та машинного навчання
(\texttt{Machine\ Learning}). Але Ви можете скористатися засобами
\texttt{R} і для простіших задач: обчислення, візуалізація даних.

Синтаксис мови програмування R є досить простим для вивчення та
використання, а широкий набір готових пакетів дозволяє використати
готові розробки для виіршення широкого спектру задач від статистичних
обчислень до навчання нейронних мереж для розпізнавання/класифікації
зображень.

Важливо відмітити, що мова програмування R є безкоштовною
(\texttt{free}) і має відкритий код (\texttt{open\ source}).

R має ряд корисних властивостей, серед яких варто виділити:

\begin{itemize}
\item
  \textbf{Візуалізація даних}. Побудова різноманітих видів графіків,
  робота з мапами, широкий спектр бібліотек та налаштувань до них.
\item
  \textbf{Повторне використання коду}. На відміну від електронних
  таблиць, що мають обмеження на кількість спостережень (наприклад, MS
  Excel), R дозволяє працювати з великими масивами даних та
  перезапускати обчислення у потрібний момент не створюючи додаткових
  копій даних.
\item
  \textbf{Машинне навчання}. R дозволяє використати для побудови,
  навчання та тестування моделей, а також оптимізації гіперпараметрів та
  відбору факторів дуже велику кількість алгоритмів. Існують також
  спеціальні пакети, що об'єднують у собі усі описані функції та
  алгоритми, наприклад, \texttt{caret} та \texttt{mlr}.
\item
  \textbf{Автоматизація}. Написаний код та проекти можна перетворити у
  готові до публікації та впровадження продукти (deployment) або
  використовувати напрацьовані алгоритми для швидкого вирішення схожих
  задач (pipeline).
\end{itemize}

Також можна виділити досить корисні фічі \textbf{Розробка
веб-застосунків} та \textbf{Звітність}, адже, використовуючи спеціальні
бібліотеки (\texttt{shiny}, \texttt{shinydashboard},
\texttt{flexdashboard}, \texttt{rmarkdown}, \texttt{knitr} тощо),
результати виконаної роботи можна ``оживити'' або сформувати ``на
льоту'' готові до презентації документи.

\hypertarget{ux43aux43eux440ux43eux442ux43aux430-ux456ux441ux442ux43eux440ux456ux44f-ux43cux438ux43eux432ux438-r}{%
\chapter{Коротка історія миови
R}\label{ux43aux43eux440ux43eux442ux43aux430-ux456ux441ux442ux43eux440ux456ux44f-ux43cux438ux43eux432ux438-r}}

Юрій Клебан\\
30/11/2022

\hfill\break

\hypertarget{ux432cux442ux430ux43dux43eux432ux43bux435ux43dux43dux44f-ux442ux430-ux43dux430ux43bux430ux448ux442ux443ux432ux430ux43dux43dux44f-r}{%
\chapter{Вcтановлення та налаштування
R}\label{ux432cux442ux430ux43dux43eux432ux43bux435ux43dux43dux44f-ux442ux430-ux43dux430ux43bux430ux448ux442ux443ux432ux430ux43dux43dux44f-r}}

Юрій Клебан\\
30/11/2022

\hfill\break

\hypertarget{ux434ux43eux43aux443ux43cux435ux43dux442ux430ux446ux456ux44f-ux442ux430-ux434ux43eux43fux43eux43cux43eux433ux430}{%
\chapter{Документація та
допомога}\label{ux434ux43eux43aux443ux43cux435ux43dux442ux430ux446ux456ux44f-ux442ux430-ux434ux43eux43fux43eux43cux43eux433ux430}}

Юрій Клебан\\
30/11/2022

\hfill\break

\hypertarget{ux440ux43eux431ux43eux442ux430-ux437-ux43fux430ux43aux435ux442ux430ux43cux438}{%
\chapter{Робота з
пакетами}\label{ux440ux43eux431ux43eux442ux430-ux437-ux43fux430ux43aux435ux442ux430ux43cux438}}

Юрій Клебан\\
30/11/2022

\hfill\break

\part{Базовий синтаксис R}

\hypertarget{ux437ux43cux456ux43dux43dux456}{%
\chapter{Змінні}\label{ux437ux43cux456ux43dux43dux456}}

Юрій Клебан\\
30/11/2022

\hfill\break

\hypertarget{ux431ux430ux437ux43eux432ux456-ux442ux438ux43fux438-ux434ux430ux43dux438ux445}{%
\chapter{Базові типи
даних}\label{ux431ux430ux437ux43eux432ux456-ux442ux438ux43fux438-ux434ux430ux43dux438ux445}}

Юрій Клебан\\
30/11/2022

\hfill\break

\hypertarget{ux43eux43fux435ux440ux430ux442ux43eux440ux438}{%
\chapter{Оператори}\label{ux43eux43fux435ux440ux430ux442ux43eux440ux438}}

Юрій Клебан\\
30/11/2022

\hfill\break

\hypertarget{ux43aux43eux440ux438ux441ux43dux456-ux43cux430ux442ux435ux43cux430ux442ux438ux447ux43dux456-ux444ux443ux43dux43aux446ux456ux457}{%
\chapter{Корисні математичні
функції}\label{ux43aux43eux440ux438ux441ux43dux456-ux43cux430ux442ux435ux43cux430ux442ux438ux447ux43dux456-ux444ux443ux43dux43aux446ux456ux457}}

Юрій Клебан\\
30/11/2022

\hfill\break

\hypertarget{ux441ux442ux432ux43eux440ux435ux43dux43dux44f-ux444ux443ux43dux43aux446ux456ux457}{%
\chapter{Створення
функції}\label{ux441ux442ux432ux43eux440ux435ux43dux43dux44f-ux444ux443ux43dux43aux446ux456ux457}}

Юрій Клебан\\
30/11/2022

\hfill\break

\part{Робота з даними у R}

\hypertarget{ux432ux435ux43aux442ux43eux440ux438}{%
\chapter{Вектори}\label{ux432ux435ux43aux442ux43eux440ux438}}

Юрій Клебан\\
30/11/2022

\hfill\break

\hypertarget{ux434ux430ux442ux430-ux444ux440ux435ux439ux43cux438}{%
\chapter{Дата-фрейми}\label{ux434ux430ux442ux430-ux444ux440ux435ux439ux43cux438}}

Юрій Клебан\\
30/11/2022

\hfill\break

\hypertarget{ux441ux43fux438ux441ux43aux438}{%
\chapter{Списки}\label{ux441ux43fux438ux441ux43aux438}}

Юрій Клебан\\
30/10/2022

\hfill\break

\bookmarksetup{startatroot}

\hypertarget{ux441ux43fux438ux441ux43eux43a-ux432ux438ux43aux43eux440ux438ux441ux442ux430ux43dux438ux445-ux434ux436ux435ux440ux435ux43b}{%
\chapter*{Список використаних
джерел}\label{ux441ux43fux438ux441ux43eux43a-ux432ux438ux43aux43eux440ux438ux441ux442ux430ux43dux438ux445-ux434ux436ux435ux440ux435ux43b}}
\addcontentsline{toc}{chapter}{Список використаних джерел}

\hypertarget{refs}{}
\begin{CSLReferences}{1}{0}
\leavevmode\vadjust pre{\hypertarget{ref-r-cran}{}}%
n.d. \emph{The Comprehensive R Archive Network}.
\url{https://cran.r-project.org/}.

\end{CSLReferences}



\end{document}
